\documentclass[twoside]{article}
\usepackage[top=1in, left=0.5in, right=0.5in, bottom=0.3in,paperheight=8.5in,paperwidth=5.5in]{geometry}
\setcounter{secnumdepth}{5}
\setcounter{tocdepth}{5}
\usepackage[english]{babel}
\usepackage{textcomp}
\usepackage{amsmath,amsthm,amsfonts,amssymb,epsfig}
\usepackage{array}
\usepackage{datetime}
\usepackage{lipsum}% http://ctan.org/pkg/lipsum
\usepackage{listings}% http://ctan.org/pkg/listings
\usepackage{spverbatim}
\usepackage{hyperref}
\usepackage{xcolor} 
\usepackage[scaled=1]{couriers}
\xdefinecolor{gray}{rgb}{0.6,0.6,0.6} 
\Urlmuskip=0mu plus 1mu\relax %needed to make long URLs break nicely
\usepackage{microtype}
\definecolor{links}{HTML}{2A1B81}
\hypersetup{colorlinks,linkcolor=,urlcolor=links}
\usepackage{graphicx}
\graphicspath{ {images/} }
\usepackage{parskip}
\usepackage{titlesec} %used for diminishing heading sizes
\titleformat{\section}{\normalfont\bfseries}{\thesection}{1em}{}
\titlespacing*{\section}{0pt}{*2}{0pt}
\titlespacing*{\subsection}{0pt}{*2}{0pt}
\usepackage[square, sort, comma, numbers]{natbib} %%uses titles for cited references
\usepackage[fit]{truncate}
\usepackage{fancyhdr}
\pagestyle{fancy}
\renewcommand{\sectionmark}[1]{\markboth{#1}{}}
\fancyhead{}
\fancyhead[OR]{\leftmark \hspace{0.1cm}  $\vert$ \hspace{0.1cm}  \thepage}
\fancyhead[EL]{\thepage \hspace{0.1cm} $\vert$ \hspace{0.1cm} \leftmark}
\fancyfoot[C]{}%hide footer 
\usepackage{setspace}
\usepackage{adjustbox} 
\usepackage[all]{nowidow} %prevent widows/orphans
\usepackage{bibentry}


%\input{common/conf_top_print.tex}  %settings for printed booklets - comment out by default, uncomment for print and comment out line above. don't save this change! "conf_top" should be default


\titleformat*{\section}{\LARGE\bfseries\sffamily}
\titleformat*{\subsection}{\Large\bfseries\sffamily}
\titleformat*{\subsubsection}{\large\bfseries\sffamily}
\titleformat*{\paragraph}{\large\bfseries\sffamily}
\titleformat*{\subparagraph}{\large\bfseries\sffamily}
\renewcommand{\familydefault}{\sfdefault} %sans-serif font


\begin{document}


%----------------------------------------------------------------------
% Definition for "lstlisting" blocks
%----------------------------------------------------------------------
% --- USAGE ---
%
% \begin{lstlisting}[style=R}
% ...
% \end{lstlisting}
%
% % \begin{lstlisting}[style=output}
% ...
% \end{lstlisting}
%----------------------------------------------------------------------

% By default, make listings all black so it's easy to spot the ones that aren't set to a style.
% This is just a debugging technique.
\lstset{backgroundcolor=\color{black}}

\lstdefinestyle{R}{
  language=R,
  frame=single,
  breaklines,
  basicstyle=\ttfamily,
  backgroundcolor=\color{white}
}

\lstdefinestyle{python}{
  language=python,
  frame=single,
  breaklines,
  basicstyle=\ttfamily,
  backgroundcolor=\color{white}
}

\definecolor{mygray}{rgb}{0.92,0.92,0.92}

\lstdefinestyle{output}{
  frame=single,
  breaklines,
  basicstyle=\ttfamily,
  backgroundcolor=\color{mygray}
}

\newcommand{\waterExampleInR} {
\textbf{Example in R} \\
}

\newcommand{\waterExampleInPython} {
\textbf{Example in Python} \\
}
 %see note for `conf_top_print.tex` above
%% TO DO: Find better templates for R and Python



%----------------------------------------------------------------------
% Definition for "lstlisting" blocks
%----------------------------------------------------------------------
% --- USAGE ---
%
% \begin{lstlisting}[style=R}
% ...
% \end{lstlisting}
%
% % \begin{lstlisting}[style=output}
% ...
% \end{lstlisting}
%----------------------------------------------------------------------

% By default, make listings all black so it's easy to spot the ones that aren't set to a style.
% This is just a debugging technique.
%\lstset{backgroundcolor=\color{black}}

% http://latexcolor.com/
\definecolor{deepblue}{rgb}{0,0,0.5}
\definecolor{deepred}{rgb}{0.6,0,0}
\definecolor{deepgreen}{rgb}{0,0.5,0}
%\definecolor{tan}{rgb}{0.98, 0.92, 0.84}  %antiquewhite
%\definecolor{r_bkgd}{rgb}{1.0, 0.92, 0.8}  %blacnedalmond
\definecolor{py_bkgd}{rgb}{0.94, 0.97, 1.0}  %aliceblue
\definecolor{ashgrey}{rgb}{0.7, 0.75, 0.71}
\definecolor{battleshipgrey}{rgb}{0.52, 0.52, 0.51}
%\definecolor{r_bkgd}{rgb}{0.97, 0.91, 0.81}  %champagne
\definecolor{r_bkgd}{rgb}{0.98, 0.92, 0.84}  %moccasin

\definecolor{Code}{rgb}{0,0,0}
\definecolor{Decorators}{rgb}{0.5,0.5,0.5}
\definecolor{Numbers}{rgb}{0.5,0,0}
\definecolor{MatchingBrackets}{rgb}{0.25,0.5,0.5}
\definecolor{Keywords}{rgb}{0,0,1}
\definecolor{self}{rgb}{0,0,0}
\definecolor{Strings}{rgb}{0,0.63,0}
\definecolor{Comments}{rgb}{0,0.63,1}
\definecolor{Backquotes}{rgb}{0,0,0}
\definecolor{Classname}{rgb}{0,0,0}
\definecolor{FunctionName}{rgb}{0,0,0}
\definecolor{Operators}{rgb}{0,0,0}
\definecolor{Background}{rgb}{0.98,0.98,0.98}

% KEYWORDS
% http://tex.stackexchange.com/questions/186092/how-can-i-delete-non-letter-keywords-such-as

\lstdefinestyle{R}{
  language={R},
  frame=single,
  breaklines,
  basicstyle=\ttfamily,
  %commentstyle=\textsl\color{Comments},% comment style
  commentstyle=\itshape\color{battleshipgrey},% comment style
  keywordstyle=\ttfamily\color{deepblue},
  numbers=left,% display line numbers on the left side 
  numberstyle=\scriptsize,% use small line numbers 
  numbersep=10pt,% space between line numbers and code
  backgroundcolor=\color{r_bkgd},
  showstringspaces=false,
  stringstyle=\color{deepgreen},
  %morekeywords={TRUE, FALSE, for, if},
  keywordstyle={\color{Keywords}\bfseries},
  keywords={TRUE, FALSE},
  deletekeywords={grid, frame, variable, model, vi, predict, file},
  otherkeywords={!,!=,~,$,*,\&,\%/\%,\%*\%,\%\%,<-,<<-},
  %morekeywords={TRUE, FALSE, list, c}
}



\lstdefinestyle{python}{
  language={Python},
  frame=single,
  breaklines,
  basicstyle=\ttfamily,
  commentstyle=\itshape\color{battleshipgrey},% comment style
  %commentstyle=\textsl,% comment style
  %keywordstyle=\ttfamily\color{deepblue},
  %keywordstyle=\color{WildStrawberry},
  numbers=left,% display line numbers on the left side 
  numberstyle=\scriptsize,% use small line numbers 
  numbersep=10pt,% space between line numbers and code
  backgroundcolor=\color{white},
  showstringspaces=false,
  stringstyle=\color{deepgreen},
  backgroundcolor=\color{py_bkgd},
  % keywords
morekeywords={import,from,class,def,for,while,if,is,in,elif,else,not,and,or,print,break,continue,return,True,False,None,access,as,,del,except,exec,finally,global,import,lambda,pass,print,raise,try,assert},
 keywordstyle={\color{Keywords}\bfseries},
}

\definecolor{mygray}{rgb}{0.92,0.92,0.92}

\lstdefinestyle{output}{
  frame=single,
  breaklines,
  basicstyle=\ttfamily,
  numbers=left,% display line numbers on the left side 
  numberstyle=\scriptsize,% use small line numbers 
  numbersep=10pt,% space between line numbers and code
  backgroundcolor=\color{mygray},
  showstringspaces=false 
}

\newcommand{\waterExampleInR} {
\textbf{Example in R} \\
}

\newcommand{\waterExampleInPython} {
\textbf{Example in Python} \\
}
  %Use this for online version


\thispagestyle{empty} %removes page number
\newgeometry{bmargin=0cm, hmargin=0cm}

\begin{center}
\textsc{\Large\bf{Machine Learning with R and H2O}}
\\
\bigskip
\line(1,0){250}  %inserts  horizontal line 

\textsc{\small {Spencer Aiello \hspace{10pt} Eric Eckstrand \hspace{10pt} Anqi Fu}}

\textsc{\small {Mark Landry  \hspace{10pt}  Jessica Lanford \hspace{10pt}  Patrick Aboyoun}}

\line(1,0){250}  %inserts  horizontal line

{\url{http://h2o.ai/resources/}}

\bigskip
\monthname \hspace{1pt}  \the\year: Third Edition 

\bigskip
\end{center}

% commenting out lines image due to print issues, but leaving in for later
%\null\vfill
%\begin{figure}[!b]
%\noindent\makebox[\textwidth]{%
%\centerline{\includegraphics[width=\paperwidth]{waves.png}}}
%\end{figure}

\newpage
\restoregeometry

\null\vfill %move next text block to lower left of new page
\thispagestyle{empty} %removes page number


{\raggedright\vfill\ 

Machine Learning with R and H2O\\

  by Spencer Aiello,\\ Eric Eckstrand, Anqi Fu, Mark Landry, \\ Jessica Lanford  \&\ Patrick Aboyoun\\
\bigskip
  Published by H2O.ai, Inc. \\
2307 Leghorn St. \\
Mountain View, CA 94043\\
\bigskip
\textcopyright \the\year \hspace{1pt} H2O.ai, Inc. All Rights Reserved. 
\bigskip

September 2015: Third Edition
\bigskip

Photos by \textcopyright H2O.ai, Inc. 
\bigskip

All copyrights belong to their respective owners.\\
While every precaution has been taken in the\\
preparation of this book, the publisher and\\
authors assume no responsibility for errors or\\
omissions, or for damages resulting from the\\
use of the information contained herein.\\
\bigskip
Printed in the United States of America. 




\newpage
\thispagestyle{empty}%remove pg#
\tableofcontents
\thispagestyle{empty}%remove pg#


\newpage


\section{Introduction}

This documentation describes how to use H2O in the R environment. More information on H2O's system and algorithms (as well as R user documentation) is available at the H2O website at {\url{http://docs.h2o.ai}}. 

R uses a REST API to connect to H2O, so you can specify the IP address and port number of the H2O instance in the R environment to use H2O in R or launch H2O from R. Data sets are not directly transmitted through the REST API. Instead, sending a command (for example, an HDFS path to import a data set) either through the browser or the REST API performs the specified task.

The data set is then assigned an identifier, which uses the .hex file type in H2O, that is used as a reference in  commands to the web server. After preparing the dataset for modeling by defining significant data and removing insignificant data, H2O creates a model representing the results of the data analysis. These models are assigned IDs that are used as references in commands. One of the most popular models for data analysis is GLM. 

GLM estimates regression models for outcomes following exponential distributions in general. In addition to the Gaussian (i.e. normal) distribution, these include binomial, gamma, Poisson, and Tweedie distributions. Each serves a different purpose, and depending on distribution and link function, can be used for prediction or classification.

This booklet demonstrates the use of H2O's implementation of GLM in an R environment. For more information on GLM, there is an additional booklet (similar to this one) available at {\url{http://h2o.ai/resources/}}. 

H2O supports Spark, YARN, and all versions of Hadoop. Hadoop is a scalable open-source file system that uses clusters for distributed storage and dataset processing. Depending on the size of your data, H2O can run on your desktop or scale using multiple nodes with Hadoop, an EC2 cluster, or S3 storage. 

H2O nodes run as JVM invocations on Hadoop nodes. For performance reasons, we recommend that you do not run an H2O node on the same hardware as the Hadoop NameNode.

Because H2O nodes run as mapper tasks in Hadoop, administrators can view them in the normal JobTracker and TaskTracker frameworks, providing process-level (i.e. JVM instance-level) visibility.

H2O helps R users make the leap from laptop-based processing to large-scale environments. Hadoop lets H2O users scale their data processing capabilities based on their current needs. Using H2O, R, and Hadoop, you can create a complete end-to-end data analysis solution. 

This document describes the four steps of data analysis with H2O: 
\begin{enumerate}

\item installing H2O
\item preparing your data for modeling (data munging)
\item creating a model using simple but powerful machine learning algorithms
\item scoring your models

\end{enumerate}

\section{What is H2O?}

H2O is fast scalable open-source machine learning and deep learning for Smarter Applications. With H2O, enterprises like PayPal, Nielsen, Cisco and others can use all of their data without sampling and get accurate predictions faster. Advanced algorithms, like Deep Learning, Boosting, and Bagging Ensembles are readily available for application designers to build smarter applications through elegant APIs. Some of our earliest customers have built powerful domain-specific predictive engines for Recommendations, Customer Churn, Propensity to Buy, Dynamic Pricing and Fraud Detection for the Insurance, Healthcare, Telecommunications, AdTech, Retail and Payment Systems.

Using in-memory compression techniques, H2O can handle billions of data rows in-memory, even with a fairly small cluster. The platform includes interfaces for R, Python, Scala, Java, JSON and Coffeescript/JavaScript, along with a built-in  web interface, Flow, that make it easier for non-engineers to stitch together complete analytic workflows. The platform was built alongside (and on top of) both Hadoop and Spark Clusters and is typically deployed within minutes.

H2O implements almost all common machine learning algorithms, such as generalized linear modeling (linear regression, logistic regression, etc.), Na\"{i}ve Bayes, principal components analysis, time series, k-means clustering and others. H2O also implements best-in-class algorithms such as Random Forest, Gradient Boosting and Deep Learning at scale. Customers can build thousands of models and compare them to get the best prediction results.

H2O is nurturing a grassroots movement of physicists, mathematicians, computer and data scientists to herald the new wave of discovery with data science. Academic researchers and Industrial data scientists collaborate closely with our team to make this possible. Stanford university giants Stephen Boyd, Trevor Hastie, Rob Tibshirani advise the H2O team to build scalable machine learning algorithms. With 100s of meetups over the past two years, H2O has become a word-of-mouth phenomenon growing amongst the data community by a 100-fold and is now used by 12,000+ users, deployed in 2000+ corporations using R, Python, Hadoop and Spark.

\textbf{Try it out}

H2O offers an R package that can be installed from CRAN, and a python package that can be installed from PyPI.

H2O can also be downloaded directly from \mbox{\url{http://h2o.ai/download}}

\textbf{Join the community}

Visit the open source community forum at \mbox{\url{https://groups.google.com/d/forum/h2ostream}}

To learn about our meetups, training sessions, hackathons, and product updates, visit \mbox{\url{http://h2o.ai}}




\section{Installation}
H2O requires Java; if you do not already have Java installed, install it from {\url{https://java.com/en/download/}} before installing H2O.

To use H2O with R, start H2O outside of R and connect to it, or launch H2O from R. However, if you launch H2O from R and close the R session, the H2O session closes as well. The H2O session directs R to the datasets and models located in H2O.

This following sections describe: 

\begin{itemize}
\item installing R 
\item installing H2O from R
%\item making a build from source code
\end{itemize}

\subsection{Installing R}

To download R:
\begin{enumerate}
\item Go to \url{http://cran.r-project.org/mirrors.html}. 
\item Select your closest local mirror. 
\item Select your operating system (Linux, OS X, or Windows). 
\item Depending on your OS, download the appropriate file, along with any required packages. 
\item When the download is complete, unzip the file and install. \\
\end{enumerate}

\subsection{Installing H2O from R}

\waterExampleInR
To load a recent H2O package from CRAN, run:
\lstinputlisting[style=R]{R_Vignette_code_examples/r_install_package.R}

{\bf{Note}}: The version of H2O in CRAN is often one release behind the current version.

For the latest recommended version, download the
latest stable H2O-3 build from the H2O download page:

\begin{enumerate}
\item Go to {\url{http://h2o.ai/download}}.
\item Choose the latest stable H2O-3 build.
\item Click the ``Install in R'' tab.
\item Copy and paste the commands into your R session.
\end{enumerate}

\waterExampleInR
After H2O is installed on your system, verify the installation completed successfully by initializing H2O:
\lstinputlisting[style=R]{R_Vignette_code_examples/r_start_help.R}

\subsection{Citation}

To cite this booklet, use the following: 

Aiello, S., Eckstrand, E., Fu, A., Landry, M., Lanford, J., and Aboyoun, P. (Aug. 2015) {\textit{Fast Scalable R with H2O}. {\url{http://h2o.ai/resources/}}. 

%Tom: Do we still want to include this section? If so, can you please make an R example that will call the latest version? 

%\subsection{Making a build from the Source Code}  
%If you are a developer who wants to make changes to the R package before building and installing it, pull the source code from Git ({\url{https://github.com/h2oai/h2o-3}}) and follow the instructions in at {\url{https://github.com/h2oai/h2o-3/blob/master/README.md}}.

%TODO
%After making the build, navigate to the top-level \texttt{h2o-3} directory using {\texttt{cd $\mathtt{\sim}$/h2o-3}}, then run the following (replacing the asterisks [*] with the version number) and install.
%\begin{lstlisting}[style=R]
%./gradlew clean 
%./gradlew build
%$ R CMD INSTALL h2o-r/R/src/contrib/h2o_****.tar.gz
%* installing to library �/Users/H2OUser/.Rlibrary�
%* installing *source* package �h2o� ...%
%** R
%** demo
%** inst
%** preparing package for lazy loading
%Creating a generic function for ...[output truncated]
%** help
%*** installing help indices
%** building package indices
%** testing if installed package can be loaded
%* DONE (h2o)
%\end{lstlisting}

\section{H2O Initialization}

This section describes how to launch H2O: 
\begin{itemize}
\item from R
\item from the command line
\item on Hadoop
\item on an EC2 cluster
\end{itemize}

\subsection{Launching from R} \label{ssec:LaunchR}

To specify the number of CPUs for the H2O session, use the \texttt{nthreads = } parameter in the \texttt{h2o.init} command. \texttt{-2} uses the CRAN default of 2 CPUs. \texttt{-1} uses 
all CPUs on the host, which is strongly recommended. To use a specific number of CPUs, enter a positive integer.

To specify the maximum amount of memory for the H2O session, use the {\texttt{max\_mem\_size}} parameter in the \texttt{h2o.init} command. The value must a multiple of 1024 greater than 2MB. Append the letter \texttt{m} or \texttt{M} to indicate megabytes, or \texttt{g} or \texttt{G} to indicate gigabytes. 

If you do not specify a value for {\texttt{max\_mem\_size}} when you run {\texttt{h2o.init}}, the default heap size of the H2O instance running on 32-bit Java is 1g. 

For best performance, the allocated memory should be 4x the size of your data, but never more than the total amount of memory on your computer. For larger data sets, we recommend running on a server or service with more memory available for computing.

H2O checks the Java version and suggests an upgrade if you are running 32-bit Java. On 64-bit Java, the heap size is 1/4 of the total memory available on the machine. 

\begin{minipage}{\textwidth}
\waterExampleInR
To launch H2O locally from R, run the following in R:
\lstinputlisting[style=R]{R_Vignette_code_examples/r_start_local.R}

\newpage
After successfully launching, R displays output similar to the following example: 
\begin{lstlisting}[style=R]
Successfully connected to http://localhost:54321
R is connected to H2O cluster:
   H2O cluster uptime:         11 minutes 35 seconds
   H2O cluster version:        2.7.0.1497 
   H2O cluster name:           H2O_started_from_R
   H2O cluster total nodes:    1
   H2O cluster total memory:   3.56 GB
   H2O cluster total cores:    8
   H2O cluster allowed cores:  8
   H2O cluster healthy:        TRUE
\end{lstlisting}
\end{minipage}

\waterExampleInR
To launch H2O locally with default initialization arguments, use the following:
\lstinputlisting[style=R]{R_Vignette_code_examples/r_init_default.R}

\waterExampleInR
To connect to an established H2O cluster (in a multi-node Hadoop environment, for example) specify the IP address and port number for the established cluster using the \texttt{ip} and \texttt{port} parameters in the \texttt{h2o.init()} command.
\medskip
\lstinputlisting[style=R]{R_Vignette_code_examples/r_connect_existing.R}

%TODO
\subsection{Launching from the Command Line}

A simple way to launch H2O from the command line is to download the H2O zip file from the H2O download page. Unzip and
launch H2O with the following:
\begin{lstlisting}[style=R]
unzip h2o-3.2.0.1-*.zip
cd h2o-3.2.0.1-*
java -jar h2o.jar
\end{lstlisting}
See the H2O Documentation for additional JVM and H2O command line options.
After launching the H2O instance, connect to it from R with {\texttt{h2o.init()}} as described above.

\subsection{Launching on Hadoop}

To launch H2O nodes and form a cluster on the Hadoop cluster, run:

\begin{lstlisting}[style=R]
hadoop jar h2odriver.jar -nodes 1 -mapperXmx 6g -output hdfsOutputDirName

\end{lstlisting}

\begin{itemize}
\item You must launch the Hadoop-specific H2O driver jar (\texttt{h2odriver.jar}) for your Hadoop distribution. Specific driver jar files are available for the following Hadoop versions:

\begin{frame}%no line table for list of 6+ items

\begin{tabular}{p{3cm}p{3cm}p{3cm}}

\begin{itemize}
     \item \texttt{cdh5.2}
     \item \texttt{cdh5.3}
     \item \texttt{cdh5.4.2}
  \end{itemize} &

\begin{itemize}
  \item \texttt{hdp2.1}
  \item  \texttt{hdp2.2}
\end{itemize} &

\begin{itemize}
 \item \texttt{mapr3.1.1}
  \item  \texttt{mapr4.0.1}
  \item  \texttt{mapr5.0}
\end{itemize}\\

\end{tabular}

\end{frame}

\item The above command launches exactly one 6g node of H2O; however,  we recommend launching the cluster with 4 times the memory of your data file.
\item{\texttt{mapperXmx}} is the mapper size or the amount of memory allocated to each node.
\item{\texttt{nodes}} is the number of nodes requested to form the cluster.
\item{\texttt{output}} is the name of the directory created each time a H2O cloud is created so it is necessary for the name to be unique each time it is launched.
\end{itemize}

\subsection{Launching on an EC2}

{\textbf{Note}}: If you would like to try out H2O on an EC2 cluster, {\url{http://play.h2o.ai}} is the easiest way to get started. H2O Play provides access to a temporary cluster managed by H2O. 
%TODO

\subsection{Checking Cluster Status}

\waterExampleInR
To check the status and health of the H2O cluster, use {\texttt{h2o.clusterInfo()}}.
\lstinputlisting[style=R]{R_Vignette_code_examples/r_cluster_info.R}

\begin{minipage}{\textwidth}

An easy-to-read summary of information about the cluster displays. 

\begin{lstlisting}[style=R]

R is connected to H2O cluster:
  H2O cluster uptime:         43 minutes 43 seconds
  H2O cluster version:        2.7.0.1497
  H2O cluster name:           H2O_started_from_R
  H2O cluster total nodes:    1
  H2O cluster total memory:   3.56 GB
  H2O cluster total cores:    8
  H2O cluster allowed cores:  8
  H2O cluster healthy:        TRUE

\end{lstlisting}

\end{minipage}

\section{Data Preparation in R}

The following section contains information about data preparation (also known as data munging) and some of the tools and methods available in H2O, as well as a data training example. 

\subsection{Notes}
\begin{itemize}

\item Although it may seem like you are manipulating the data in R, once the data has been passed to H2O, all data munging occurs in the H2O instance. The information is passed to R through JSON APIs, so some functions may not have another method. 
\item You are limited by the total amount of memory allocated to the H2O instance, not by R's ability to handle data. To process large data sets, make sure to allocate enough memory. For more information, refer to {\textbf{\nameref{ssec:LaunchR}}}. 
\item You can manipulate datasets with thousands of factor levels using H2O in R, so if you ask H2O to display a table in R with information from high cardinality factors, the results may overwhelm R`s capacity. 
\item To manipulate data in R and not in H2O, use {\texttt{as.data.frame()}}, {\texttt{as.h2o()}}, and {\texttt{str()}}. \begin{itemize}
\item {\texttt{as.data.frame()}} converts an H2O data frame into an R data frame. If your request exceeds the amount of data supported by R, the R session will crash. If possible, we recommend only taking subsets of the entire data set (the necessary data columns or rows) instead of the whole data set. 
\item {\texttt{as.h2o()}} transfers data from R to the H2O instance. For successful data transfer, we recommend confirming enough memory is allocated to the H2O instance.
\item {\texttt{str.H2OFrame()}} returns the elements of the new object to confirm that the data transferred correctly. It$'$s a good way to verify there were no data loss or conversion issues. %% is this still supported? can't find it in the R package doc...
%%TODO - replaced with possible new command; confirm (JL)
\end{itemize}
\end{itemize}

%TODO: still ok? if so, add to common

%\subsection{Tools and Methods}

%The following section describes some of the tools and methods available in H2O for data preparation. 
%\begin{itemize}
%\item {\textbf{Data Profiling}}: Quickly summarize the shape of your dataset to avoid bias or missing information before you start building your model. Missing data, zero values, text, and a visual distribution of the data are visualized automatically upon data ingestion. 
%\item {\textbf{Summary Statistics}}: Visualize your data with summary statistics to get the mean, standard deviation, min, max, or quantile (for numeric columns) or cardinality and counts (for enum columns), and a preview of the data set. 
%\item {\textbf{Aggregate, Filter, Bin, and Derive Columns}}: Build unique views with Group functions, Filtering, Binning, and Derived Columns. 
%\item {\textbf{Slice, Log Transform, and Anonymize}}: Normalize and partition to get your data into the right shape for modeling, and anonymize to remove confidential information. 
%\item {\textbf{Variable Creation}}: Highly customizable variable value creation to hone in on the key data characteristics to model. 
%\item {\textbf{PCA}}: Principal Component Analysis makes feature selection easy with a simple interface and standard input values to reduce the many dimensions in your dataset into key components. 
%\item {\textbf{Training and Validation Sampling Plan}}: Design a random or stratified sampling plan to generate data sets for model training and scoring. 
%\end{itemize}

\subsection{Demo: Creating Aggregates from Split Data}

The following section depicts an example of creating aggregates for data training using {\texttt{ddply()}}. Using this method, you can split your dataset and apply a function to the subsets.

\waterExampleInR
To apply a user-specified function to each subset of an H2O dataset and combine the results, use {\texttt{ddply()}}, with the name of the H2O object, the variable name, and the function in the parentheses.
\lstinputlisting[style=R]{R_Vignette_code_examples/r_ddply.R}

\section{Models}
%TODO: Another candidate for "common"

The following section describes the features and functions of some common models available in H2O.  For more information about running these models in R using H2O, refer to {\textbf{\nameref{sec:RunModel}}}. 

%%mal H2O supports the following models: Deep Learning (DL), Generalized Linear Models (GLM), Gradient Boosted Regression (GBM), Na\"{i}ve Bayes (NB), Random Forest (RF), K-Means, Principal Components Analysis (PCA) and Generalized Low Rank Model (GLRM).

\begin{minipage}{\textwidth}
H2O supports the following models:  


\begin{frame}%no line table for list of 6+ items

\begin{tabular}{p{5.5cm}p{5.5cm}}

\begin{itemize}
 \item Deep Learning
  \item Na\"{i}ve Bayes
  \item Principal Components Analysis (PCA)
  \item K-means
\end{itemize} &

\begin{itemize}
     \item Generalized Linear Models (GLM) 
     \item Gradient Boosted Regression (GBM)
%  \item Generalized Low Rank Model (GLRM) Supported?
  \item Distributed Random Forest (DRF)
\end{itemize}

\end{tabular}

\end{frame}
\end{minipage}

The list is growing quickly, so check \url{www.h2o.ai} to see the latest additions. The following list describes some common model types and features. 

\subsection{Supervised Learning}

{\textbf{Generalized Linear Models (GLM)}}: Provides flexible generalization of ordinary linear regression for response variables with error distribution models other than a Gaussian (normal) distribution. GLM unifies various other statistical models, including Poisson, linear, logistic, and others when using $\ell_1$ and $\ell_2$ regularization.

%%mal {\textbf{Decision trees}}: Used in RF; a decision support tool that uses a tree-like graph or model of decisions and their possible consequences.
{\textbf{Distributed Random Forest}}: Averages multiple decision trees, each created on different random samples of rows and columns. It is easy to use, non-linear, and provides feedback on the importance of each predictor in the model, making it one of the most robust algorithms for noisy data.

{\textbf{Gradient Boosting (GBM)}}: Produces a prediction model in the form of an ensemble of weak prediction models. It builds the model in a stage-wise fashion and is generalized by allowing an arbitrary differentiable loss function. It is one of the most powerful methods available today.

{\textbf{Deep Learning}}: Models high-level abstractions in data by using non-linear transformations in a layer-by-layer method. Deep learning is an example of supervised learning, which can use unlabeled data that other algorithms cannot.

{\textbf{Na\"{i}ve Bayes}}: Generates a probabilistic classifier that assumes the value of a particular feature is unrelated to the presence or absence of any other feature, given the class variable. It is often used in text categorization.

\subsection{Unsupervised Learning}

{\textbf{K-Means}}: Reveals groups or clusters of data points for segmentation. It clusters observations into $k$-number of points with the nearest mean.

{\textbf{Anomaly Detection}}: Identifies the outliers in your data by invoking the deep learning autoencoder, a powerful pattern recognition model.

\subsection{Modeling Constructs}

{\textbf{Grid Search}}: Performs standard hyper-parameter optimization to simplify model configuration. %%should be supported in latest version, but adding note in case

After creating a model, use it to make predictions. For more information about predictions, refer to {\textbf{\nameref{ssec:Predictions}}}.

%%mal: this feels too lengthy for a modeling example here; perhaps it would be better to have entire work flows at the end?

\section{Demo: GLM}

The following demo demonstrates how to:

\begin{enumerate}
\item  Import a file
\item Define significant data
\item View data
\item Create testing and training sets using sampling
\item Define the model
\item Display the results
\end{enumerate}

\waterExampleInR
\lstinputlisting[style=R]{R_Vignette_code_examples/r_glm_demo.R}

\section{Data Manipulation in R}

The following section describes some common R commands. For a complete command list, including parameters, refer to {\url{http://h2o-release.s3.amazonaws.com/h2o/latest_stable_Rdoc.html}}.
For additional help within R's Help tab, precede the command with a question mark (for example, {\texttt{?h2o}}) for suggested commands containing the search terms. For more information on a command, precede the command with two question marks ({\texttt{??h2o}}). 

\subsection{Importing Files}

\waterExampleInR
The H2O package consolidates all of the various supported import functions using {\texttt{h2o.importFile()}}. % {\texttt{h2o.importFolder}} and {\texttt{h2o.importHDFS}} still work, these functions are deprecated and should be updated to {\texttt{h2o.importFile()}}.
There are a few ways to import files shown in the following examples:
\lstinputlisting[style=R]{R_Vignette_code_examples/r_import_file.R}

\subsection{Uploading Files}

\waterExampleInR
To upload a file in a directory local to your H2O instance, use {\texttt{h2o.importFile()}}.  {\texttt{h2o.uploadFile()}}  uploads data local to your H2O instance as well as uploading data local to your R session. In the parentheses, specify the H2O reference object in R and the complete URL or normalized file path for the file.
\lstinputlisting[style=R]{R_Vignette_code_examples/r_upload_file.R}


\subsection{Finding Factors}

\waterExampleInR
To determine if any column contains categorical data (also known as a factor), use {\texttt{h2o.anyFactor()}}, with the R reference object in the parentheses.
\lstinputlisting[style=R]{R_Vignette_code_examples/r_factors.R}

\subsection{Converting to Factors}

\waterExampleInR
To convert an integer into a non-ordered factor (also called an enum or categorical), use {\texttt{as.factor()}} with the name of the R reference object in parentheses, followed by the number of the column to convert in brackets.
\lstinputlisting[style=R]{R_Vignette_code_examples/r_to_factors.R}

\subsection{Converting Data Frames}

\normalsize

\waterExampleInR
To convert an H2O parsed data object into an R data frame that can be manipulated using R commands, use {\texttt{as.data.frame()}} with the name of the R reference object in the parentheses.
{\textbf{Caution}}: While this can be very useful, be careful when using this command to convert H2O parsed data objects. H2O can easily handle data sets that are often too large to be handled equivalently well in R.
\lstinputlisting[style=R]{R_Vignette_code_examples/r_as_data_frame.R}

\subsection{Transferring Data Frames}

\waterExampleInR
To transfer a data frame from the R environment to the H2O instance, use  {\texttt{as.h2o()}}. In the parentheses, specify the object in the R environment to convert to an H2O object. Optionally, include the name of the destination frame in H2O. Precede the destination frame name with {\texttt{destination\_frame = }} and enclose the name in quotes as in the following example.
\lstinputlisting[style=R]{R_Vignette_code_examples/r_as_h2o.R}


\subsection{Renaming Data Frames}

\waterExampleInR
To rename a dataframe on the server running H2O for a data set manipulated in R, use {\texttt{h2o.assign()}}. In the following example, the prostate data set was uploaded to the H2O instance and the data was manipulated to remove outliers. {\texttt{h2o.assign()}} saves the new data set on the H2O server so it can be analyzed using H2O without overwriting the original data set.
\lstinputlisting[style=R]{R_Vignette_code_examples/r_assign.R}


\subsection{Viewing Column Names}

\waterExampleInR
To view a list of the column names in the data set, use {\texttt{colnames()}} or {\texttt{names()}} with the name of the R reference object in the parentheses.
\lstinputlisting[style=R]{R_Vignette_code_examples/r_colnames.R}

\subsection{Getting Minimum and Maximum Values}

\waterExampleInR
To view the maximum values for the real-valued columns in a data set, use {\texttt{max()}} with the name of the R reference object in the parentheses.
To obtain the minimum values for the real-valued columns in a data set, use {\texttt{min()}} with the name of the R reference object in the parentheses.
\lstinputlisting[style=R]{R_Vignette_code_examples/r_min_max.R}


\subsection{Getting Quantiles}

\waterExampleInR
To request quantiles for an H2O parsed data set, use {\texttt{quantile()}} with the name of the R reference object in the parentheses.
To request a quantile for a single numerical column, use  {\texttt{quantile(ReferenceObject\$ColumnName)}},  where {\texttt{ReferenceObject}} represents the R reference object name and \\{\texttt{ColumnName}} represents the name of the specified column.
When you request for a full parsed data set consisting of a single column, {\texttt{quantile()}} displays a matrix with quantile information for the data set.
\lstinputlisting[style=R]{R_Vignette_code_examples/r_quantiles.R}

\newpage
\subsection{Summarizing Data}

\waterExampleInR
To generate a summary similar to the one in R for each of the columns in the data set, use {\texttt{summary()}} with the name of the R reference object in the parentheses.
For continuous real functions, this produces a summary that includes information on quartiles, min, max, and mean.
For factors, this produces information about counts of elements within each factor level.
\lstinputlisting[style=R]{R_Vignette_code_examples/r_summary.R}


\subsection{Summarizing Data in a Table}

\waterExampleInR
To summarize data, use {\texttt{h2o.table()}}. Because H2O can handle larger data sets, it is possible to generate tables that are larger than R`s capacity, so use caution when executing this command.
To summarize multiple columns, use {\texttt{head(h2o.table (ObjectName[, c(ColumnNumber,ColumnNumber)]))}}, where {\texttt{ObjectName}} is the name of the object in R and {\texttt{ColumnNumber}} is the number of the column.
\lstinputlisting[style=R]{R_Vignette_code_examples/r_table.R}

\subsection{Generating Random Numbers}

\waterExampleInR
To append a column of random numbers to an H2O data frame for testing/training data splits that are used for analysis and validation in H2O, use {\texttt{h2o.runif()}} with the name of the R reference object in the parentheses. This method is best for customized frame splitting; otherwise, use {\texttt{h2o.splitFrame()}}. However, {\texttt{h2o.runif()}} is not as fast or stable as {\texttt{h2o.splitFrame()}}.
\lstinputlisting[style=R]{R_Vignette_code_examples/r_runif.R}
 

\subsection{Splitting Frames}

\waterExampleInR
To generate two subsets (according to specified ratios) from an existing H2O data set for testing/training, use {\texttt{h2o.splitFrame()}}, which returns contiguous sections of the data without random sampling.
\lstinputlisting[style=R]{R_Vignette_code_examples/r_splitframe.R}

\subsection{Getting Frames}

\waterExampleInR
To create a reference object to the data frame in H2O, use {\texttt{h2o.getFrame()}}. This is helpful for  switching between the web UI and the R API or for multiple users accessing the same H2O instance. The following example assumes prostate.hex is in the key-value (KV) store.
\lstinputlisting[style=R]{R_Vignette_code_examples/r_getframe.R}


\subsection{Getting Models}

\waterExampleInR
To create a reference object for the model in H2O, use {\texttt{h2o.getModel()}}. This is helpful for  users that alternate between the web UI and the R API or multiple users accessing the same H2O instance. The following example assumes a GBM with the ID \texttt{GBM\_8e4591a9b413407b983d73fbd9eb44cf} is in the key-value (KV) store.
\lstinputlisting[style=R]{R_Vignette_code_examples/r_getmodel.R}


\newpage
\subsection{Listing H2O Objects}

\waterExampleInR
To generate a list of all H2O objects generated during a session and each object’s size in bytes, use {\texttt{h2o.ls()}}.
\lstinputlisting[style=R]{R_Vignette_code_examples/r_ls.R}

\normalsize 

\subsection{Removing H2O Objects}

\waterExampleInR
To remove an H2O object on the server associated with the object in the R environment, use {\texttt{h2o.rm()}}. For optimal performance, we recommend removing the object from the R environment as well using {\texttt{remove()}}, with the name of the object in the parentheses. If you do not specify an R environment, then the current environment is used.
\lstinputlisting[style=R]{R_Vignette_code_examples/r_rm.R}


\subsection{Adding Functions}

\waterExampleInR
User-defined functions no longer need to be added explicitly to the H2O instance. An R function can be defined and executed against an H2OFrame.
\lstinputlisting[style=R]{R_Vignette_code_examples/r_addfunc.R}



\section{Running Models}
\label{sec:RunModel}

This section describes how to run the following model types:

\begin{itemize}
\item Gradient Boosted Models (GBM)
\item Generalized Linear Models (GLM)
\item K-Means 
\item Principal Components Analysis (PCA)

\end{itemize}
as well as how to generate predictions. 


\subsection{Gradient Boosted Models (GBM)}

\waterExampleInR
To generate gradient boosted models for developing forward-learning ensembles, use {\texttt{h2o.gbm()}}.  In the parentheses, define \texttt{x} (the predictor variable vector), \texttt{y} (the integer or categorical response variable), the distribution type (multinomial is the default, gaussian is used for regression), and the name of the H2OParsedData object.
For more information, use {\texttt{help(h2o.gbm)}}.
\lstinputlisting[style=R]{R_Vignette_code_examples/r_gbm.R}

\waterExampleInR
To generate a classification model that uses labels, use {\texttt{distribution= "multinomial"}}:
\lstinputlisting[style=R]{R_Vignette_code_examples/r_gbm_multinomial.R}



\subsection{Generalized Linear Models (GLM)}


\waterExampleInR
Generalized linear models (GLM) are some of the most commonly-used models for many types of data analysis use cases. While some data can be analyzed using general linear models, general linear models may not be as accurate if the variables are more complex. For example, if the dependent variable has a non-continuous distribution or if the effect of the predictors is not linear, generalized linear models will produce more accurate results than general linear models.
Generalized Linear Models (GLM) estimate regression models for outcomes following exponential distributions in general. In addition to the Gaussian (i.e. normal) distribution, these include Poisson, binomial, gamma and Tweedie distributions. Each serves a different purpose, and depending on distribution and link function choice, it can be used either for prediction or classification.
H2O's GLM algorithm fits the generalized linear model with elastic net penalties. The model fitting computation is distributed, extremely fast,and scales extremely well for models with a limited number ($\sim$ low thousands) of predictors with non-zero coefficients. The algorithm can compute models for a single value of a penalty argument or the full regularization path, similar to glmnet. It can compute Gaussian (linear), logistic, Poisson, and gamma regression models.
To generate a generalized linear model for developing linear models for exponential distributions, use {\texttt{h2o.glm()}}. You can apply regularization to the model by adjusting the lambda and alpha parameters.
For more information, use {\texttt{help(h2o.glm)}}.
\lstinputlisting[style=R]{R_Vignette_code_examples/r_glm2.R}



\subsection{K-Means}

\waterExampleInR
To generate a K-Means model for data characterization, use {\texttt{h2o.kmeans()}}. This algorithm does not rely on a dependent variable. For more information, use {\texttt{help(h2o.kmeans)}}.
\lstinputlisting[style=R]{R_Vignette_code_examples/r_kmeans.R}



\subsection{Principal Components Analysis (PCA)}

\waterExampleInR
To map a set of variables onto a subspace using linear transformations, use {\texttt{h2o.prcomp()}}. This is the first step in Principal Components Regression. For more information, use {\texttt{help(h2o.prcomp)}}.
\lstinputlisting[style=R]{R_Vignette_code_examples/r_pca.R}



\subsection{Predictions}
\label{ssec:Predictions}

The following section describes some of the prediction methods available in H2O. 

{\textbf{Predict}}: Generate outcomes of a data set with any model. Predict with GLM, GBM, Decision Trees or Deep Learning models.

{\textbf{Confusion Matrix}}: Visualize the performance of an algorithm in a table to understand how a model performs.

{\textbf{Area Under Curve (AUC)}}: A graphical plot to visualize the performance of a model by its sensitivity, true positives and false positives to select the best model.

{\textbf{Hit Ratio}}: A classification matrix to visualize the ratio of the number of correctly classified and incorrectly classified cases.

{\textbf{PCA Score}}: Determine how well your feature selection fits a particular model.

{\textbf{Multi-Model Scoring}}: Compare and contrast multiple models on a data set to find the best performer to deploy into production. 

\waterExampleInR
To apply an H2O model to a holdout set for predictions based on model results, use {\texttt{h2o.predict()}}.
In the following example, H2O generates a model and then displays the predictions for that model.
For classification, the predict column is the model's discrete prediction, based on maximum F1 by default; the individual class probabilities are the remaining columns in the data frame.
It is common to utilize the p1 column for binary classification, if a raw probability is desired.
\lstinputlisting[style=R]{R_Vignette_code_examples/r_predict.R}


\newpage
\section{Appendix: Commands} \label{Appendix} 

The following section lists some common R commands by function and a brief description of each command. 

\subsection {Data Set Operations}
{\bgroup\obeylines %needed for formatting

{\emph{Data Import/Export}}
{\texttt{h2o.downloadCSV}}: Download a H2O dataset to a CSV file on local disk.
{\texttt{h2o.exportFile}}: Export H2O Data Frame to a file.
{\texttt{h2o.importFile}}: Import a file from the local path and parse it. 
{\texttt{h2o.parseRaw}}: Parse a raw data file.  
{\texttt{h2o.uploadFile}}: Upload a file from the local drive and parse it. 
\medskip
{\emph{Native R to H2O Coercion}}\smallskip
 {\texttt{as.h2o}}: Convert an R object to an H2O object. 
\medskip
\emph{H2O to Native R Coercion}\par
{\texttt{as.data.frame}}: Check if an object is a data frame, or coerce it if possible. 
\medskip
\emph{Data Generation}\par
{\texttt{h2o.createFrame}}: Create an H2O data frame, with optional randomization. 
{\texttt{h2o.runif}}: Produce a vector of random uniform numbers. 
{\texttt{h2o.interaction}}: Create interaction terms between categorical features of an H2O Frame. 
\medskip
\emph{Data Sampling/Splitting}\par
{\texttt{h2o.splitFrame}}: Split an existing H2O data set according to user-specified ratios. 
\newpage
\emph{Missing Data Handling}\par
{\texttt{h2o.impute}}: Impute a column of data using the mean, median, or mode. 
{\texttt{h2o.insertMissingValues}}: Replaces a user-specified fraction of entries in a H2O dataset with missing values. 
}
\subsection{General Data Operations}
{\bgroup\obeylines %needed for formatting
\emph{Subscripting example to pull pieces from data object.} 
\begin{lstlisting}[style=R]
  x[j]  ## note: chooses column J, not row J
  x[i, j]
  x[[i]]
  x$name 
  x[i] <- value
  x[i, j, ...] <- value
  x[[i]] <- value
  x$i <- value
\end{lstlisting}

\emph{Subsetting} 
\medskip
{\texttt{head, tail}}: Return the First or Last Part of an Object 
\medskip
\emph{Concatenation}\par
 {\texttt{c}}: Combine Values into a Vector or List \\ %%still supported? couldn't find in R doc
 {\texttt{h2o.cbind}}: Take a sequence of H2O datasets and combine them by column.\par
\medskip
{\emph{Data Attributes}}\par
{\texttt{colnames}}: Return column names for a parsed H2O data object.  
{\texttt{colnames$<$-}}: Retrieve or set the row or column names of a matrix-like object. 
{\texttt{names}}: Get the name of an object.  
{\texttt{names$<$-}}: Set the name of an object.  
{\texttt{dim}}: Retrieve the dimension of an object.  
{\texttt{length}}: Get the length of vectors (including lists) and factors.  
{\texttt{nrow}}: Return a count of the number of rows in an H2OParsedData object.  
{\texttt{ncol}}: Return a count of the number of columns in an H2OParsedData object. 
{\texttt{h2o.anyFactor}}: Check if an H2O parsed data object has any categorical data columns.  
{\texttt{is.factor}}: Check if a given column contains categorical data. 
\medskip
{\emph{Data Type Coercion}}\par
{\texttt{as.factor}}: Convert a column from numeric to factor. 
{\texttt{as.Date}}: Converts a column from factor to date. 
}

\subsection{Methods from Group Generics}
{\bgroup\obeylines %needed for formatting
\medskip
{\emph{Math (H2O)}}\par
{\texttt{abs}}: Compute the absolute value of x.  
{\texttt{sign}}: Return a vector with the signs of the corresponding elements of x (the sign of a real number is 1, 0, or -1 if the number is positive, zero, or negative, respectively).  
{\texttt{sqrt}}: Computes the principal square root of x, $\sqrt{x}$. 
{\texttt{ceiling}}: Take a single numeric argument x and return a numeric vector containing the smallest integers not less than the corresponding elements of x.  
{\texttt{floor}}: Take a single numeric argument x and return a numeric vector containing the largest integers not greater than the corresponding elements of x.  
{\texttt{trunc}}: Take a single numeric argument x and return a numeric vector containing the integers formed by truncating the values in x toward 0.  
{\texttt{log}}: Compute logarithms (by default, natural logarithms).  
{\texttt{exp}}: Compute the exponential function. 
\medskip
{\emph{Math (generic)}}\par
{\texttt{cummax}}: Display a vector of the cumulative maxima of the elements of the argument. 
{\texttt{cummin}}: Display a vector of the cumulative minima of the elements of the argument. 
{\texttt{cumprod}}: Display a vector of the cumulative products of the elements of the argument.  
{\texttt{cumsum}}: Display a vector of the cumulative sums of the elements of the argument.   
{\texttt{log10}}: Compute common (i.e., base 10) logarithms  
{\texttt{log2}}: Compute binary (i.e., base 2) logarithms.  
{\texttt{log1p}}: Compute log(1+x) accurately also for $|x|${}\textless{}\textless{} 1. 
{\texttt{acos}}: Compute the trigonometric arc-cosine.  
{\texttt{acosh}}: Compute the hyperbolic arc-cosine.  
{\texttt{asin}}: Compute the trigonometric arc-sine. 
{\texttt{asinh}}: Compute the hyperbolic arc-sine. 
{\texttt{atan}}: Compute the trigonometric arc-tangent.  
{\texttt{atanh}}: Compute the hyperbolic arc-tangent. 
{\texttt{expm1}}: Compute exp(x) - 1 accurately also for $|x|$\textless{}\textless{} 1. 
{\texttt{cos}}: Compute the trigonometric cosine. 
{\texttt{cosh}}: Compute the hyperbolic cosine.  
{\texttt{cospi}}: Compute the trigonometric two-argument arc-cosine.  
{\texttt{sin}}: Compute the trigonometric sine.  
{\texttt{sinh}}: Compute the hyperbolic sine.  
{\texttt{sinpi}}: Compute the trigonometric two-argument arc-sine.  
{\texttt{tan}}: Compute the trigonometric tangent. 
{\texttt{tanh}}: Compute the hyperbolic tangent. 
{\texttt{tanpi}}: Compute the trigonometric two-argument arc-tangent.  
{\texttt{gamma}}: Display the gamma function $\gamma{x}$  
{\texttt{lgamma}}: Display the natural logarithm of the absolute value of the gamma function.  
{\texttt{digamma}}: Display the first derivative of the logarithm of the gamma function.  
{\texttt{trigamma}}: Display the second derivative of the logarithm of the gamma function. 
\medskip
\emph{Math2 (H2O)}\par
{\texttt{round}}: Round the values to the specified number of decimal places. The default is 0.  
{\texttt{signif}}: Round the values to the specified number of significant digits. 

\emph{Summary (H2O)}\par
{\texttt{max}}: Display the maximum of all the input arguments.  
{\texttt{min}}: Display the minimum of all the input arguments.  
{\texttt{range}}: Display a vector containing the minimum and maximum of all the given arguments.  
{\texttt{sum}}: Calculate the sum of all the values present in its arguments. 

\medskip
\emph{Summary (generic)}\par
{\texttt{prod}}: Display the product of all values present in its arguments.  
{\texttt{any}}: Given a set of logical vectors, determine if at least one of the values is true.  
{\texttt{all}}: Given a set of logical vectors, determine if all of the values are true.
}
\subsection{Other Aggregations}
{\bgroup\obeylines %needed for formatting
\medskip
\emph{Non-Group Generic Summaries}\par
{\texttt{mean}}: Generic function for the (trimmed) arithmetic mean.  
{\texttt{sd}}: Calculate the standard deviation of a column of continuous real valued data.  
{\texttt{var}}: Compute the variance of x. 
{\texttt{summary}}: Produce result summaries of the results of various model fitting functions.  
{\texttt{quantile}}: Obtain and display quantiles for H2O parsed data. 

\medskip
\emph{Row / Column Aggregation}\par
{\texttt{apply}}: Apply a function over an H2O parsed data object (an array). 

\medskip
\emph{Group By Aggregation}\par
{\texttt{h2o.ddply}}: Split H2O dataset, apply a function, and display results. 
{\texttt{h2o.group\_by}}: Apply an aggregate function to each group of an H2O dataset. 

\medskip
\emph{Tabulation}\par
{\texttt{h2o.table}}: Use the cross-classifying factors to build a table of counts at each combination of factor levels.
}
\subsection{Data Munging}\par
{\bgroup\obeylines %needed for formatting
\medskip
\emph{General Column Manipulations}\par
{\texttt{is.na}}: Display missing elements.   

\medskip
\emph{Element Index Selection}\par 
{\texttt{h2o.which}}: Display the row numbers for which the condition is true. 

\medskip
\emph{Conditional Element Value Selection}\par 
{\texttt{h2o.ifelse}}: Apply conditional statements to numeric vectors in H2O parsed data objects. 

\medskip
\emph{Numeric Column Manipulations}\par
{\texttt{h2o.cut}}: Convert H2O Numeric Data to Factor.  

\medskip
\emph{Character Column Manipulations}\par
{\texttt{h2o.strsplit}}: Splits the given factor column on the input split.  
{\texttt{h2o.tolower}}: Change the elements of a character vector to lower case.  
{\texttt{h2o.toupper}}: Change the elements of a character vector to lower case.  
{\texttt{h2o.trim}}: Remove leading and trailing white space. 
{\texttt{h2o.gsub}}: Match a pattern \& replace all instances of the matched pattern with the replacement string globally.  
{\texttt{h2o.sub}}: Match a pattern \& replace the first instance of the matched pattern with the replacement string. 

\medskip
\emph{Factor Level Manipulations}\par
{\texttt{h2o.levels}}: Display a list of the unique values found in a column of categorical data.  

\medskip
\emph{Date Manipulations}\par
{\texttt{h2o.month}}: Convert the entries of a H2OParsedData object from milliseconds to months (on a 0 to 11 scale).  
{\texttt{h2o.year}}: Convert the entries of a H2OParsedData object from milliseconds to years, indexed starting from 1900. 

\emph{Matrix Operations}\par
{\texttt{\%$*$\%}}}: Multiply two matrices, if they are conformable. 
{\texttt{t}}: Given a matrix or data.frame x, t returns the transpose of x. 
}
\subsection{Data Modeling}
{\bgroup\obeylines %needed for formatting
\medskip
\emph{Model Training: Supervised Learning}\par
{\texttt{h2o.deeplearning}}: Perform Deep Learning neural networks on an\\ H2OParsedData object. 
{\texttt{h2o.gbm}}: Build gradient boosted classification trees and gradient boosted regression trees on a parsed data set.  
{\texttt{h2o.glm}}: Fit a generalized linear model, specified by a response variable, a set of predictors, and a description of the error distribution.  
{\texttt{h2o.naiveBayes}}: Build gradient boosted classification trees and gradient boosted regression trees on a parsed data set. 
{\texttt{h2o.prcomp}}: Perform principal components analysis on the given data set.  
{\texttt{h2o.randomForest}}: Perform random forest classification on a data set. 

\medskip
\emph{Model Training: Unsupervised Learning}\par
{\texttt{h2o.anomaly}}: Detect anomalies in a H2O dataset using a H2O deep learning model with auto-encoding. 
{\texttt{h2o.deepfeatures}}: Extract the non-linear features from a H2O dataset using a H2O deep learning model. 
{\texttt{h2o.kmeans}}: Perform k-means clustering on a data set.  

\medskip
\emph{Grid Search}\par
{\texttt{h2o.grid}}: Efficient method to build multiple models with different hyperparameters.  

\medskip
\emph{Model Scoring}\par
{\texttt{h2o.predict}}: Obtain predictions from various fitted H2O model objects. 

\medskip
\emph{Classification Model Helpers}\par
{\texttt{h2o.accuracy}}: Get the between cluster sum of squares. 
{\texttt{h2o.auc}}: Retrieve the AUC (area under ROC curve). 
{\texttt{h2o.confusionMatrix}}: Display prediction errors for classification data from a column of predicted responses and a column of actual (reference) responses in H2O. 
{\texttt{h2o.hit\_ratio\_table}}: Retrieve the Hit Ratios. If {\texttt{train}}, {\texttt{valid}}, and {\texttt{xval}} parameters are FALSE (default), then the training Hit Ratios value is returned. If more than one parameter is set to TRUE, then a named list
of Hit Ratio tables are returned, where the names are {\texttt{train}}, {\texttt{valid}}, or {\texttt{xval}}. 
{\texttt{h2o.performance}}: Evaluate the predictive performance of a model via various measures. 

\medskip
\emph{Regression Model Helper}\par
{\texttt{h2o.mse}}: Display the mean squared error calculated from a column of predicted responses and a column of actual (reference) responses in H2O. 

\medskip
\emph{Clustering Model Helper}\par
{\texttt{h2o.betweenss}}: Get the between cluster sum of squares. 
{\texttt{h2o.centers}}: Retrieve the Model Centers. 
}
\subsection{H2O Cluster Operations}
{\bgroup\obeylines %needed for formatting
\medskip
\emph{H2O Key Value Store Access}\par
{\texttt{h2o.assign}}: Assign H2O hex.keys to objects in their R environment. 
{\texttt{h2o.getFrame}}: Get a reference to an existing H2O data set.  
{\texttt{h2o.getModel}}: Get a reference to an existing H2O model.  
{\texttt{h2o.ls: }}Display a list of object keys in the running instance of H2O.  
{\texttt{h2o.rm}}: Remove H2O objects from the server where the instance of H2O is running, but does not remove it from the R environment. 

\medskip
\emph{H2O Object Serialization}\par
{\texttt{h2o.loadModel}}: Load an H2OModel object from disk. 
{\texttt{h2o.saveModel}}: Save an H2OModel object to disk to be loaded back into H2O using {\texttt{h2o.loadModel}}. 

\emph{H2O Cluster Connection}\par 
{\texttt{h2o.init (nthreads = -1)}}: Connect to a running H2O instance using all CPUs on the host and check the local H2O R package is the correct version. 
{\texttt{h2o.shutdown}}: Shut down the specified H2O instance. All data on the server will be lost! 

\medskip
\emph{H2O Load Balancing}\par
{\texttt{h2o.rebalance}}: Rebalance (repartition) an existing H2O data set into given number of chunks (per Vec), for load-balancing across multiple threads or nodes.  %%still supported? couldn't find in R doc

\medskip
\emph{H2O Cluster Information}\par
{\texttt{h2o.clusterInfo}}: Display the name, version, uptime, total nodes, total memory, total cores and health of a cluster running H2O. 
{\texttt{h2o.clusterStatus}}: Retrieve information on the status of the cluster running H2O. 

\medskip
\emph{H2O Logging}\par
{\texttt{h2o.clearLog}}: Clear all H2O R command and error response logs from the local disk.  
{\texttt{h2o.downloadAllLogs}}: Download all H2O log files to the local disk. 
{\texttt{h2o.logAndEcho}}: Write a message to the H2O Java log file and echo it back.   
{\texttt{h2o.openLog}}: Open existing logs of H2O R POST commands and error responses on the local disk. 
{\texttt{h2o.getLogPath}}: Get the file path for the H2O R command and error response logs. 
{\texttt{h2o.startLogging}}: Begin logging H2O R POST commands and error responses.  
{\texttt{h2o.stopLogging}}: Stop logging H2O R POST commands and error responses. 

\medskip
\emph{H2O String Manipulation}\par
{\texttt{h2o.gsub}}: String global substitution (all occurrences). 
{\texttt{h2o.strsplit}}: String Split. 
{\texttt{h2o.sub}}: String substitution (first occurrence). 
{\texttt{h2o.tolower}}: Convert characters to lower case. 
{\texttt{h2o.toupper}}: Convert characters to upper case. 
{\texttt{h2o.trim}}: Trim spaces. 
}
%\egroup %needed for formatting



\newpage

\newpage

\section{Authors}

\textbf{Spencer Aiello}

Spencer comes from an unconventional background. After studying Physics and Math as an undergraduate at UCSC, he came to San Francisco to continue his education, earning his MS in Analytics from USF.  Spencer has worked on a number of projects related to analytics in R, Python, Java, and SQL. At H2O, he works primarily on the R front-end and the backend Java R-interpreter.

\textbf{Eric Eckstrand}

Eric is a Quality and Performance Hacker at H2O.ai. Eric has formal education in computer science and systems engineering. Prior to joining H2O, Eric was a submariner in the US Navy. His roles included Reactor Controls Assistant and Communications Officer (USS Pittsburgh SSN-720) and Submarine Operations Officer (Destroyer Squadron One). 

\textbf{Anqi Fu}

Anqi is a data hacker at H2O.ai. After completing her undergraduate at University of Maryland: College Park, she attended Stanford University. While there she earned her master’s degree in Economics, and a second master’s degree in Statistics. Her interests include machine learning and optimization.

\textbf{Mark Landry}

Mark Landry is a competition data scientist and product manager at H2O. He enjoys testing ideas in Kaggle competitions, where he is ranked in the top 100 in the world (top 0.03$\%$) and well-trained in getting quick solutions to iterate over. Interests are multi-model architectures and helping the world make fewer models that perform worse than the mean.

\textbf{Jessica Lanford}

Jessica is a word hacker and seasoned technical communicator at H2O.ai. She brings our product to life by documenting the many features and functionality of H2O. Having worked for some of the top companies in technology including Dell, AT$\&$T, and Lam Research, she is an expert at translating complex ideas to digestible articles.


\textbf{Patrick Aboyoun}

Patrick is a former math and data hacker at H2O.ai who has made a career out of creating and delivering software and training for data scientists, particularly those who love R.  Patrick received an M.S. in Statistics from the University of Washington and a B.S. in Statistics from Carnegie Mellon University.

\newpage
\section{References}

\bibliographystyle{plainnat}  

\nobibliography{bibliography}

\bibentry{h2osite}
%
\bibentry{h2odocs}
%
\bibentry{h2ogithubrepo}
%
\bibentry{h2odatasets}
%
\bibentry{h2ojira}
%
\bibentry{rdocs}
%
\bibentry{rlanguage}
%
\bibentry{stream}
%
\bibentry{rensemble}

\bibentry{h2o_R_package}

\end{document}
